%!TEX root = demo-paper.tex
\vspace{-2mm}
\section{Demo Walkthrough}
\label{sec:demo-walkthrough}
 
We propose to demonstrate the functionality of \SeeDB\ through hands-on
interaction with a variety of datasets. Our goals are two fold: (1) demonstrate
the utility of \SeeDB\ in surfacing interesting trends for a query
and (2) demonstrate that we can return high quality views efficiently for
a range of datasets. We will use four different datasets in our demonstration:

\begin{denselist}
  \item {\bf Store Orders dataset}~\cite{superstore}: This dataset is
    often used by Tableau~\cite{tableau} as a canonical dataset for
    business intelligence applications. It consists of information
    about orders placed in a store including products, prices, ship
    dates, geographical information, and profits. Interesting trends in
    this dataset have been very well studied, and attendees will use
    \SeeDB\ to quickly re-identify these trends. 
    %This dataset will also
    %enable us to demonstrate how \SeeDB can correctly deal with
    %numeric, categorical, and geographic data.
  \item {\bf Election Contribution dataset}~\cite{election_data}: This
  is an example of a dataset typically analyzed by
    non-expert data analysts like journalists or historians. With this
    dataset, we demonstrate how non-experts can use \SeeDB\ to quickly
    arrive at interesting visualizations.
  \item {\bf Medical dataset}~\cite{mimic}: This real-world dataset exemplifies
  a dataset that a clinical researcher might use. The schema of the dataset is
  significantly complex and it is of larger size.  
    \item {\bf Synthetic data:} We provide a set of synthetic datasets with
    varying sizes, number of attributes, and data distributions to help
    attendees evaluate \SeeDB\ performance on diverse datasets.
\end{denselist}

\stitle {Scenario 1: Demonstrating Utility.} Attendees are provided with three
diverse, real-world datasets to explore using \SeeDB. For each dataset,
attendees can issue ad-hoc or pre-formulated queries to \SeeDB. \SeeDB\ will
then intelligently explore the view space and optimize query execution to return the
most interesting visualizations with low latency. Attendees can examine the
returned queries visually, view the associated metadata, and perform
drill-downs. To aid the evaluation of visualizations, the demo system will 
be configured to also show the user ``bad'' views (views with low utility) that were not selected
by \SeeDB.
Similarly, we provide pre-selected queries (and
previously known information about their results) to allow attendees to
confirm that \SeeDB\ does indeed reproduce known information about these
queries. Attendees will also be able to experiment with a
variety of distance metrics for computing utility and observe the effects on the
resulting views.

% \stitle{Demonstrating Utility:} To show the utility of \SeeDB\ in a real-world
% scenario, we will provide conference attendees three diverse datasets that they
% can explore and interact with. Attendees can pose ad-hoc or pre-selected queries
% on various datasets and evaluate the visualizations returned. The
% evaluation is based on whether the visualizations surface ``interesting''
% aspects of the queried data and whether the right visualizations have been
% selected. To aid the evaluation of visualizations, the demo version of \SeeDB\
% will have the option of showing ``bad'' visualizations too, i.e. visualizations
% that were predicted to have low utility.  The attendees will also have the option of
% trying various utility metrics as described in Section
% \ref{sec:problem_statement}. The demo datasets will include:

\stitle{Scenario 2: Demonstrating Performance and Optimizations.} This scenario
will use an enhanced user interface and synthetic datasets mentioned above.
Attendees will be able to easily experiment with a range of synthetic datasets and input
queries by adjusting various ``knobs'' such as data size, number of attributes, and
data distribution. In addition, attendees will also be able to select the
optimizations that \SeeDB\ applies and observe the effect on response times and
accuracy.

Thus, through our demonstration of \SeeDB\, we seek to illustrate that (a) it is
possible to automate labor-intensive parts of data analysis, (b) aggregate
and grouping-based views are a powerful means to identify interesting trends
in data, and (c) the right set of optimizations can enable real-time data
analysis of large datasets.

\eat{
We propose to demonstrate the functionality of the SeeDB system by means of
analyzing three diverse datasets of practical importance. Users will be able to
explore each of these datasets in real-time by using SeeDB to formulate
queries and find interesting trends in the underlying dataset. Specifically, we
will use the following datasets for demonstration purposes:

\begin{itemize}
  \item {\bf Store Orders dataset}: This is a canonical dataset used in business
  intelligence applications. It consists of information about orders placed in a
  store including products, prices, ship dates, geographical information,
  profits etc. The dataset is well known for its interesting trends and
  richness of various data types. It will show off SeeDB capabilities to
  correctly identify diverse trends in the data and the ability to deal with
  numeric, categorical, time series and geographic data.
  \item {\bf Election Contribution dataset}: This dataset is a great example of
  the kind of data and analysis that must be done by potential users like
  journalists who are not data analysts by trade but often need to find
  interesting trends in datasets. As a result, this use case will help the
  audience guage the intuitiveness of the user interface, ease of use and fast
  response times.
  \item {\bf Medical dataset:} This dataset is an example of a dataset that a
  researcher (here, a clinical researcher) might use over the course of his/her
  work. This data has a schema that is more complex than the the election
  or store one, and is of larger size too. Since this data is usually analyzed
  by experts, in addition to fast provision of insights, the user also cares about
  flexiblity and ``expert'' operations on this data such as statistical
  information, accuracy of visualizations, drill-downs etc.
\end{itemize}

We envision the demonstration workflow to be as follows: The user selects one of
the three dataset from above for analysis. He/she formulates a selection query
using the SeeDB query builder or by using ready-made queries (e.g. selecting
outliers in a column etc.) and submits the query to SeeDB. SeeDB then searches
through the entire space of possible views using techniques and heuristics
described in Section \ref{optimizations} and returns the top-{\it k} views it
considers most interesting. The SeeDB frontend then visualizes the top-{it k} views and
presents them to the user. The user can interact with each of these views and
perform further exploration through operations such as drill-downs (graphically
selecting subsets of data), comparisons of multiple views etc. The user will
also be able to experiment with the effect of choosing different utility metrics
and optimization strategies described in Section \ref{}.
}
