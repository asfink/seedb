%!TEX root=document.tex


Data analysts operating on large volumes of data often rely on visualizations to
interpret the results of queries.
However, finding the right visualization given a query of interest is a
laborious and time-consuming task.
We propose \SeeDB, a system to partially automate this task:
given a query, \SeeDB\ intelligently explores the space of all possible
visualizations, evaluates promising visualizations, and automatically recommends
the most ``interesting'' or ``useful'' visualizations to the analyst.
In this paper, we present two implementations of \SeeDB\ which make very
different design choices: the first leverages existing database systems as the
backend and aggressively optimizes queries to get the highest performance; the
second is a proof-of-concept implementation that has been engineered from
scratch and, without the constraints of the DBMS API, can overcome many of the
limitations of the first design.
For both implementations, we develop and evaluate a suite of optimizations that
leverage the underlying system properties.
Our experiments on a range of real world and synthetic datasets demonstrate
that our optimizations speed up processing by up to 8-20X in the DBMS-backed
execution engine.
Similarly, our custom engine can achieve a 10-fold speedup by aggressively
pruning low-utility views. We further show that pruning does not adversely
affect the accuracy of views returned.
With the DBMS optimizations and pruning heuristics, we demonstrate that
\SeeDB can be used to recommend interesting views in near-interactive time
scales.
% Our end-to-end experiments also demonstrate that that \SeeDB\ can be used to
% find interesting visualizations in interactive time scales.
% Finally, we present the results of a user study that evaluates our method for
% interesting visualizations, the relative quality of different distance metrics
% and the \SeeDB\ system.
