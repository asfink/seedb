%!TEX root=document.tex


% Data analysts operating on large volumes of data often rely on visualizations to
% interpret the results of queries.
% However, finding the right visualization given a query of interest is a
% laborious and time-consuming task.
% We propose \SeeDB, a visualization recommendation 
% engine that aims to partially automate this task:
% given a query, \SeeDB\ intelligently explores the space of all possible
% visualizations, evaluates promising visualizations, and automatically recommends
% the most ``interesting'' or ``useful'' visualizations to the analyst.
% We present two types of optimizations for \SeeDB: sharing-based optimizations,
% that try to share as much computation as possible,
% and pruning-based optimizations, that try to avoid as much computation
% on not-so-useful visualizations as possible,
% and show that these optimizations may be applied to implementations of \SeeDB over
% a relational row store database and  a column store database.
% We demonstrate how our optimizations lead to {\em multiple
% orders of magnitude speedup} on \SeeDB on both types of databases,
% while not significantly impacting accuracy.
% We further demonstrate via a user study that \SeeDB returns
% interesting and useful visualizations. 
% Overall, \SeeDB represents significant progress 
%  towards the challenging and important task of automated visualization generation.


% In this paper, we present two implementations of \SeeDB\ which make very
% different design choices: the first leverages existing database systems as the
% backend and aggressively optimizes queries to get the highest performance; the
% second is a proof-of-concept implementation that has been engineered from
% scratch and, without the constraints of the DBMS API, can overcome many of the
% limitations of the first design.
% For both implementations, we develop and evaluate a suite of optimizations that
% leverage the underlying system properties.
% Our experiments on a range of real world and synthetic datasets demonstrate
% that our optimizations speed up processing by up to 8-20X in the DBMS-backed
% execution engine.
% Similarly, our custom engine can achieve a 10-fold speedup by aggressively
% pruning low-utility views. We further show that pruning does not adversely
% affect the accuracy of views returned.
% With the DBMS optimizations and pruning heuristics, we demonstrate that
% \SeeDB can be used to recommend interesting views in near-interactive time
% scales.
% Our end-to-end experiments also demonstrate that that \SeeDB\ can be used to
% find interesting visualizations in interactive time scales.
% Finally, we present the results of a user study that evaluates our method for
% interesting visualizations, the relative quality of different distance metrics
% and the \SeeDB\ system.

Data analysts often build visualizations
as the first step in their analytical workflow.
However, when working with high-dimensional datasets, identifying visualizations
that show relevant or desired trends in data can be laborious.
%The need to manually specify visualizations one-at-a-time hampers rapid analysis and exploration.
We propose \SeeDB, a visualization recommendation engine to facilitate fast 
visual analysis: 
given a subset of data to be studied, \SeeDB intelligently explores the 
space of visualizations, evaluates promising visualizations for trends, and 
recommends those it 
deems most ``useful'' or ``interesting''. % for that data relative to the rest.
The two major obstacles in recommending interesting visualizations are (a) {\em scale}:
dealing with a large number of candidate visualizations and evaluating all of them in parallel, while
responding within interactive time scales, and (b) {\em utility}: identifying 
an appropriate metric for assessing interestingness of visualizations.
For the former, \SeeDB introduces {\em pruning optimizations} to quickly 
identify high-utility visualizations and {\em sharing optimizations} to maximize
sharing of computation across visualizations.
For the latter, as a first step, % in crafting a sophisticated utility scoring function, 
we adopt a 
deviation-based metric for visualization 
utility, while indicating how we may be able to generalize it to other factors
influencing utility.
We implement \SeeDB as a middleware layer that can run on top of any DBMS. 
Our experiments show that our framework can identify interesting visualizations with high accuracy. 
Our optimizations lead to 
{\em multiple orders of magnitude speedup} on relational row and column stores and provide
recommendations at interactive time scales.
Finally, we demonstrate via a user study the effectiveness of our deviation-based utility metric
and the value of recommendations in supporting visual analytics. 
