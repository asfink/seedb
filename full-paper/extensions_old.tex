%!TEX root=document.tex
\techreport{
\section{Convergence of Utility Metrics}
\label{sec:convergence}

\stitle{Convergence of Sampling Algorithms.}
We now demonstrate how,
when applied to the utility function $U$ as described
in Section~\ref{sec:problem_statement},
sampling algorithms 
converge to the correct value of $U$ beyond a certain number of samples.

For now, we focus on the case where the visualization $V_i$
corresponds to the AVG aggregate. 
Similar results can be shown for the SUM and STD
aggregates. 
Unfortunately, MAX and MIN are not amenable to sampling-based
optimizations, as is traditionally well-known in the approximate
query processing literature~\cite{wavelets,dbo}.

Additionally, we focus on the case when $S$ is defined to be
$\ell_2$, i.e., the Euclidean distance metric. 
Once again, similar results can be shown for other distance metrics,
such as the $\ell_1$, the Earth Movers Distance metric, or
the Jenson-Shannon metric.

We reproduce the utility equation here:
$ U (V_i) = S ( P[V_i (D_Q)], P[V_i (D)] )$.
Here, $P$ refers to the probability distribution
of either the target visualization
or the comparison visualization.
$P$ is represented as a normalized vector
whose entries sum up to $1$.
We have the following: 
\papertext{(The proof can be found in the extended technical report~\cite{seedb-tr}.)}
\begin{lemma}
Let the target and comparison visualizations
both have $m$ groups.
Let $\hat{U}$ denote our estimate of $U$ based on a uniformly random sample 
across all $m$ groups. 
Then, as the number of samples tends to $\infty$, $\hat{U} \rightarrow U$
with probability $1-\delta$, for as small $\delta$ as needed.
\end{lemma}
\papertext{At a high level, the proof
involves repeated applications of Hoeffding's inequality to
upper and lower-bound $\hat{U}$ within $U$ along with terms 
that tend to $0$ as the number of samples increases.}
}
\techreport{
\begin{proof} (Sketch)
Let us say the estimated average for the target visualization
for each of the groups is $\hat{t}_i$,
and the estimated averages for the comparison visualization
for each of the groups is $\hat{c}_i$.
We further define $\hat{t}$ (respectively $\hat{c}$) to be the estimated sum of
averages for the target visualization (respectively comparison visualization).
We let the true values for each of these quantities be the same variables without
the hats.
Then, it can be shown that that $U$ evaluates to:
$$\hat{U} = \frac{\sum{\hc_i^2}}{\hc^2} +  \frac{\sum{\htt_i^2}}{\htt^2} - 2 \frac{\sum{\htt_i \hc_i}}{\hc\htt}$$


Now we informally describe the steps of our proof:
say we sample enough to get $\hc$ within $\epsilon$ of $c$, with a high enough probability,
and we sample enough to get $\htt$ within $\epsilon$ of $t$, with a high enough probability.
Then, we have 
\begin{align*}
\hat{U} & \geq \frac{\sum{\hc_i^2}}{(c + \epsilon)^2} +  \frac{\sum{\htt_i^2}}{(t + \epsilon)^2} - 2 \frac{\sum{\htt_i \hc_i}}{(c - \epsilon) (t - \epsilon)} \\
& \geq \frac{\sum{\hc_i^2}}{c^2} (1 - \epsilon ) +  \frac{\sum{\htt_i^2}}{t ^2} (1-\epsilon) - 2 \frac{\sum{\htt_i \hc_i}}{(c - \epsilon) (t - \epsilon)} \\
& \geq \frac{\sum{\hc_i^2}}{c^2} (1 - \epsilon ) +  \frac{\sum{\htt_i^2}}{t ^2} (1-\epsilon) - 2 \frac{\sum{\htt_i \hc_i}}{ct}(1 + \epsilon^2 + \epsilon) 
\end{align*}
Similarly, if we have sampled enough to get the $\hc_i$ and the $\htt_i$ within $\gamma$ close 
of their actual values, we will have:
\begin{align*}
\hat{U} \geq &  \frac{\sum{c_i^2}}{c^2} (1 + f(\gamma)) (1 - \epsilon ) +   \frac{\sum{t_i^2}}{t ^2}  (1 + f(\gamma)) (1-\epsilon) \\ & - 2 \frac{\sum{t_i c_i}}{ct}(1 + h(\gamma))(1 + \epsilon^2 + \epsilon) 
\end{align*}
where $f(.)$ and $h(.)$ are small polynomial functions.
Thus,
we will have sandwiched $\hat{U}$ from the bottom by $U-\rho$,
and similarly by $U + \rho'$ from the top.
$\rho, \rho'$ will be polynomials that depend on $\epsilon$ and $\gamma$.
Now, we will use the Hoeffding's inequality for the last step of the proof.
Hoeffding's inequality, when
applied to a collection of $n$ i.i.d. random variables,
whose sum is represented by $X$, gives us:
\begin{equation}
Pr (|X - E[X]| \geq t) \leq 2 e^{-\frac{2 n t^2}{c^2}}
\end{equation}
where $c$ is a bound on the range. 
If we set the right hand side to some $\delta$, and set $t = \epsilon$,
we have
$$ \epsilon = \sqrt{\frac{1}{2 n} \ln \frac{2}{\delta}}$$
and therefore, as $n$ tends to $\infty$, $\epsilon$ tends to $0$,
for fixed values of $\delta$.
The same holds true for $t = \gamma$.
Thus, $\hat{U}$ will tend to $U$ as the number of samples
increases to infinity.
\end{proof}

It is in fact also possible to explicitly derive a number of samples 
such that $\hat{U}$ is close to $U$ within a certain error bound
and with a certain probability. 
}



\section{Extensions}
\label{sec:discussion}

\subsection{Extension to other Utility Metrics}
\label{sec:other_utility_metrics}

\subsection{Comparison of Different distance Metrics for Deviation}
\label{sec:diff_dist_metrics}
\mpv{something goes here}

\subsection{Metrics other than deviation}

% In this section we prove that our sampling 
% algorithms converge, and then describe our generalized
% distance metric.

In Section \ref{sec:introduction}, we introduced a set of {\it dimensions} that determine the quality of a visualization.
In subsequent sections, we developed a system that used metadata, data distributions, and the input query
to recommend visualizations based on deviation.
In this section, we discuss how our deviation metric can encompass a variety of distribution-based metrics and how we can augment our system to take other quality dimensions into account.

First, consider an extension of the utility function $U (V_i)$ denoted as $f_D (V_i)\ =\ w_d \times S_d + w_h \times S_h + w_l  \times S_l$.
Here, the first component corresponds to the utility definition from Section~\ref{sec:problem_statement}
so that $S_d = S ( P[V_i (D_Q)],$ $P[V_i (D)] )$ takes into account the deviation between the data selected by the query and the comparison data.
We can use the same distance metric $S$ to also capture utility based on historical context. 
For instance, let $S_h = S ( P[V_i (D_Q)], $ $P[V_i (D_C)] )$ where
$P[V_i (D_C)]$ refers to the typical value of the distribution 
$P[V_i (D_Q)]$, given historical data.
For instance, when $V_i$ refers to sales of a particular chair in 
Boston, $V_i(D_C)$ could be sales of that chair in Boston for the past 10 years.
% This measure would then allow us to identify whether the value for particular sales is 
% deviating significantly from past patterns and is therefore interesting.
Finally, we can also use the deviation-based metric to capture local trends.
For instance, let $S_l = S ( P[V_i (D_Q)], P'[V_i (D_Q)] )$ where
$P'[V_i (D_Q)]$ refers to the distribution $P$, but shifted slightly.
For instance, if the sensor readings in the last five minutes differ greatly
for current readings, the utility of $V_i$ would be high.
This component can capture
the amount of rapid local changes that have happened
within the distribution corresponding to $P[V_i (D_Q)]$.
A combination of the above deviation-based metrics provides a rich model to capture a large
portion of distribution-based utility.

Next, we turn to the question of incorporating other recommendation dimensions into our utility 
metric.
Recall from Section \ref{sec:introduction} that along with metadata, query and distribution (the
dimensions supported by our metric), other recommendation dimensions include aesthetics, user
preferences and semantics.
Consider the following form of a generalized utility function:
$$ U (V_i) = f_{MA}(V_i) \times f_P (V_i) \times f_D (V_i)$$

Let $f_D (V_i)$ be the utility function measuring distribution-based utility of $V_i$.
We can then augment $f_D (V_i)$ with $f_{MA}(V_i)$ and $f_P (V_i)$ capturing metadata and semantics, and user preferences respectively.
For instance, $f_{MA}(V_i)$ can capture best-practices about visualization and output a utility value accordingly.
Similarly, let $f_P (V_i)$ be a function that models the users' preference towards seeing visualization $V_i$. 
This function can take into account past user history at both the individual and global levels.
For instance, if the analyst typically looks at sales over time, $f_P (V_i)$ for sales over time
may be high.
Similarly, if sales is a popular attribute across all data analysts, $f_P (V_i)$ could be large for
a $V_i$ that depicts sales.
We can also think of $f_P (V_i)$ as capturing in a limited way semantics that can be mined from previous user interactions.
We note however that semantics is one dimension that an automated recommendation system will not be able to capture fully.

In terms of the visualization recommendations, we find that $f_{MA}(V_i)$ and $P (V_i)$ are independent
of the data distribution.
Once a user poses a query, we can merely reuse previously computed values for these functions while
making the recommendation.
For \SeeDB, these functions amount merely to constants in the utility function $U (V_i)$ that would assign weights to each view.
Thus, in this form, we find that \SeeDB can easily incorporate other recommendation dimensions into
the utility metric without any changes to the \SeeDB framework.

% Let $f_{MA}(V_i)$ be a function that captures utility of visualization $V_i$ along the dimensions of metadata and aesthetics (e.g. a heatmap of city temperatures would have a high utility value while a bar chart of the same data would have low utility).

% Finally, we turn to $f_D (V_i)$, a function capturing utility of a visualization based on distributions in
% the data.
% As mentioned before, there is a large host of means to capture utility based on data distributions.
% In this work, we focused on a particular component of $D (V_i)$, namely the one measuring deviation between
% the query distribution and the background distribution.
% However, we can easily use the deviation model to capture various aspects of distribution-based
% utility.
% Imagine that $D (V_i)$ is decomposed into $w_1 \times S_1 + w_2 \times S_2 + w_3  \times S_3$.



% The third component can into account local changes.
% ,
% and only makes sense when the $x$ axis is an ordinal attribute
% (such as time, or location, or rating).

% By adopting a (well-behaved) utility metric $U (V_i)$ of the form described above, 
% we can use \SeeDB to make recommendations based not only on distribution but also take other dimensions into account.

% For well-behaved distance metrics $U (V_i)$ of the form described above, we use the existing \SeeDB framework to make recommendations incorporating other distribution based utility functions, and different recommendation dimensions including user preferences and semantics.

% The taxonomy in Section \ref{sec:introduction} introduced various desirable qualities in a utility
% mertric for visualization recommendation. These were metadata, aesthetics, query, data distribution, 
% user preferences, and semantics.
% Current systems like Tableau and Spotfire only use metadata and aesthetics to make recommendations about
% visualizations.
% In this work, we focused on developing a system that incorporates metadata, query and data distribution into
% its recommendations. 
% % We ignore aesthetics for the moment since decisions about aesthetics can be made once the system has chosen 
% % the particular view of the data to be visualized (e.g. individual records, order statistics etc).
% Specifically, our utility metric from Section \ref{sec:problem_statement} captured a particular instance of 
% turning the query and data distribution into a visualization utility.


% Consider the following generalized distance metric.
% $M(V_i)$ denotes the utility of a visualization based
% on metadata, 
% $A(V_i)$ denotes utility based on aesthetics (e.g. whether a bar chart is more appropriate or a pie chart),
% $P (V_i)$ denotes utility based on user preferences for seeing the specific 
% visualization attributes.

% $S (V_i)$ corresponds to semantics or external knowledge unknown to the system.
% $D (V_i)$ corresponds to utility of a visualization based on data distribution.
% Each of these functions in turn have sub-components that capture various means of measuring utility
% along that dimension.
% We do not explicitly model the impact of the input query because it is an essential component of all the above
% functions.

% $$ U (V_i) = f_{MA}(V_i) \times P (V_i) \times S (V_i) \times D (V_i)$$



% In this manner, the generalized distribution metric can take into
% account (a) data and query, as before
% (b) user preferences, (c) context,
% and (d) local changes.
% Naturally, the generalized distribution metric that
% we have proposed is only a starting point for further exploration
% into visualization recommendation metrics.
% We expect any metric to take into account all the features
% that we have listed, and more.



% In Section~\ref{sec:problem_statement}, we described our
% distance metric that depended only on the deviation of the data selected
% by the query $D_Q$ from the background data $D$.
% This metric made sense for the simple case
% when an analyst is studying a dataset for the first time.
% We now describe how we can generalize this distance
% metric to take into account more information,
% in order to provide more useful visualization recommendations.
% Our generalized metric is as follows:

% $$ U (V_i) = f(V_i) \times (w_1\times S_1 + w_2 \times S_2 + w_3 \times S_3)$$
% Our generalized metric has three distribution
% distance components, $S_1, S_2,$ and $S_3$,
% weighted by suitable weights $w_1, w_2,$ and $w_3$.








