%!TEX root=document.tex

\subsection{Sharing-based Optimizations}
\label{sec:sharing_opt}
% In this section, we focus on minimizing total execution time
% by reducing the
% total number of queries issued to the database
% and by reducing the total number of scans of the underlying table.
% Sharing computation in our setting is a special case of the general problem
% of multi-query optimization~\cite{DBLP:journals/tods/Sellis88}; we discuss the 
% relationship in more detail in Section~\ref{sec:related_work}.

% Each visualization evaluated by \SeeDB translates into two 
% aggregation queries on the DBMS, one for the target view and one for the comparison view.
% The queries corresponding to each visualization are
%  similar; they scan the same underlying data,  
% differing only in the  attributes used for grouping and aggregation.
% As a result, there are opportunities to reduce the number of queries and passes over the data 
% by merging and batching queries.

In the naive implementation, each visualization is translated into view
queries that get executed independently on the DBMS.
However, for a particular user input, the queries evaluated by \SeeDB
are very similar: they scan the same underlying data and differ only in the 
attributes used for grouping and aggregation.
This presents opportunities to intelligently merge and batch queries, reducing 
the number of queries issued to the database and, in turn, minimizing the number 
of scans of the underlying data.
Sharing computation in our setting is a special case of the general problem
of multi-query optimization~\cite{DBLP:journals/tods/Sellis88}; we discuss the 
relationship in more detail in Section~\ref{sec:related_work}.
Specifically, we apply the following optimizations:

\stitle{Combine Multiple Aggregates}: Aggregate view queries 
with the same group-by attribute can be 
rewritten as a single query with multiple aggregations. 
Therefore, instead of views $(a_1$, $m_1$, $f_1)$, $(a_1$, $m_2$, $f_2)$ 
\ldots $(a_1$, $m_k$, $f_k)$, each requiring execution of two queries, we combine 
these views into a single view $(a_1, \{m_1, m_2\ldots m_k\}$, $\{f_1, f_2\ldots f_k\})$ 
requiring execution of just two queries.
We have found that there is minimal to no impact on latency 
for combining aggregates in both row and column stores. 

% (two queries each for) views $(a_1$, $m_1$, $f_1)$, $(a_1$, $m_2$, $f_2)$ 
% \ldots $(a_1$, $m_k$, $f_k)$, \SeeDB combines them into a single view 
% $(a_1, \{m_1, m_2\ldots m_k\}$, $\{f_1, f_2\ldots f_k\})$ which can be executed via just
% two queries.  

\stitle {Combine Multiple GROUP BYs}:
After applying our multiple aggregates optimization, \SeeDB is left with a number of 
queries with multiple aggregations but only single-attribute groupings.
These queries can be further combined to take advantage of multi-attribute grouping.
However, unlike combining multiple aggregates, the addition of a grouping attribute can 
dramatically increase the number of groups that must be maintained and (possibly)
lead to slower overall performance for large number of groups.

We claim (and verify in Section \ref{sec:experiments}) that grouping can benefit 
performance so long as memory utilization for grouping stays under a threshold.
Memory utilization is, in turn, proportional to the number of distinct groups 
present in a query.
If a set of attributes $a_1$\ldots$a_m$ are used for grouping, the upperbound on the 
number of distinct groups is given by $\prod_{i=1}^m |a_i|$. 
Given a memory budget $\mathcal{S}$, the challenge is now to determine the optimal grouping
of attributes such that each group respects the memory budget.
\vspace{-5pt}
\begin{problem}[Optimal Grouping]
Given memory budget $\mathcal{S}$ and a set of dimension attributes $A$ = \{$a_1$\ldots$a_n$\}, 
divide the dimension attributes in $A$ into groups $A_1, \ldots, A_l$ (where $A_i$ $\subseteq$ $A$ 
and $\bigcup A_i$=$A$) such that if a query $Q$ groups the table by any $A_i$, 
the memory utilization for $Q$ does not exceed $\mathcal{S}$.
\vspace{-5pt}
\end{problem}

Notice that the above problem is isomorphic to the NP-Hard {\em bin-packing} problem~\cite{garey}.
If we let each dimension attribute
$a_i$ correspond to an item in the bin-packing problem with weight $\log (|a_i|)$,
and set the bin size to be $\log \mathcal{S}$,
then packing items into bins is identical to finding groups $A_1, \ldots, A_l$,
such that the estimated memory utilization for any group is below $\mathcal{S}$.
We use the standard first-fit algorithm~\cite{first-fit} to find the optimal
grouping of dimension attributes.


\stitle{Combine target and reference view query}:
Since the target and reference views differ only in the subset of data on 
which the query is executed, \SeeDB rewrites these two view queries as
one. For instance, if the target and reference view queries are $Q1$ and $Q2$
respectively, they can be combined into a single query $Q3$.
\vspace{-5pt}
\begin{align*} 
Q1 = &{\tt SELECT \ } a, f(m) \ \ {\tt FROM} \  D\  {\tt WHERE \ \ x\ <\ 10\
GROUP \ \ BY} \ a \\
Q2 = &{\tt SELECT \ } a, f(m) \ \ {\tt FROM} \  D\  {\tt GROUP \ \ BY} \ a \\
Q3 = &{\tt SELECT \ } a, f(m), {\tt CASE\ IF\ x\ <\ 10\ THEN\ 1\ ELSE\ 0\
END}\\ 
&as\ g1,\ 1\ as\ g2 \ \ {\tt FROM} \ D\ {\tt GROUP \ \ BY} \ a,\ g1,\ g2
\end{align*}



\stitle {Parallel Query Execution}:
  \SeeDB executes multiple view queries in parallel as these queries can often
 share buffer pool pages, reducing disk access times. 
  However, the precise number of parallel queries needs to be tuned taking into account 
  buffer pool contention, locking, and cache line contention, among other factors~\cite{Postgres_wiki}.  

\techreport{
\stitle{Other Optimizations}: 
To further speedup processing, \SeeDB can also pre-compute results for 
static views (e.g. reference views on full tables) or operate on
pre-computed data samples.  Such optimizations are orthogonal to the
problem of efficiently evaluating a large number of views, which we must address
even in the presence of pre-computation or sampling.
}