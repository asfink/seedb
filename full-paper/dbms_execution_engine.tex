%!TEX root=document.tex
In the next two sections, we discuss our implementations of the \SeeDB\
execution engine. 
Input to the \SeeDB\ engine is a set of view stubs (triples of the form
$(a, m, f)$, where $a$ is the dimension attribute, $m$ is the measure,
and $f$ is the aggregation function) and the output is the top-$k$ views 
with the highest utility.
%The goals of the \SeeDB\ engine are two fold:
%(1) to efficiently compute the utility of a large number of views, and 
%(2) to accurately rank views in order of
%utility to find the $k$ views with highest utility.
% As discussed in the architecture overview, we explore two distinct
% implementations of the \SeeDB\ execution engine.

\section{DBMS-backed Execution Engine}
\label{sec:dbms-exec-engine}

Our first design implements the \SeeDB\ engine as a wrapper on top of a
traditional database system.
We explore this design because it enables \SeeDB\ to be easily integrated 
with visualization systems that use DBMSs for data storage (e.g. Tableau, 
Spotfire etc).
In fact, this design is oblivious to the particular data store as long as
it supports a JDBC-like interface.
On the downside, this design limits \SeeDB\ to using the API
exposed by the DBMS; essentially, 
\SeeDB\ can only open/close connections to the
database and execute one or more SQL queries. 
%Since we have no control over how queries are executed or access to intermediate
%results, 
As a result, our optimizations minimize total execution time by minimizing the
total number of queries and total scans of the underlying table.

The basic procedure for identifying top-$k$ views is the following: generate 
SQL queries for the target and comparison view for each potential visualization, 
execute each query to compute aggregate distributions; and compute distance between
distributions to pick the $k$ visualizations with the highest utility. 
% Given a set of view stubs provided to the execution engine, conceptually,
% the basic approach proceeds as follows:
% (1) for each view, we generate SQL queries for the target and
% comparison view---recall that these are aggregate queries, where
% the target view is the aggregate query corresponding 
% to the visualization being considered, and the comparison
% view is the aggregate query that is being compared against,
% for the purpose of utility computation;
% see Section \ref{sec:problem_statement}, 
% (2) we execute each view query independently on the DBMS, 
% (3) the results of each query (i.e. the aggregated values) are processed and
% normalized to compute the target and comparison distributions, 
% (4) we compute utility for each view 
% and select the top-$k$ views with the largest utility,
% which are then sent to the frontend.
In a table with $a$ dimensions, $m$ measures, and $f$ aggregation functions, 
$2\times f \times a \times  m$ queries must be executed independently, a 
solution that takes prohibitively long (over a minute in row
stores, Section \ref{sec:experiments}). \mpv{Talk about grouping sets somewhere}.
% This strawman approach is clearly inefficient since it examines every possible view and 
% executes each view query independently.
We can apply a set of well-studied optimization techniques from OLAP cube computation and
multi-query optimization to improve upon this strawman approach.
Specifically, we apply the following set of four optimizations to speed up \SeeDB\ processing
within a database.

\stitle{Combine Multiple Aggregates}: All view queries with the same group-by attribute are 
rewritten as a single, 
combined view query. So instead of executing
queries for views $(a_1$, $m_1$, $f_1)$, $(a_1$, $m_2$, $f_2)$ \ldots $(a_1$, $m_k$, $f_k)$
independently, we combine them into a single view represented by
$(a_1, \{m_1, m_2\ldots m_k\}$, $\{f_1, f_2\ldots f_k\})$. 

\stitle{Combine target and comparison view query}:
Since the target view and comparison views only differ in the subset of data
that the query is executed on, we rewrite these two view queries as
one. For instance, the target and comparison view queries $Q1$ and $Q2$
shown below, can be combined into a single query $Q3$.
\vspace{-5pt}
\begin{align*} 
Q1 = &{\tt SELECT \ } a, f(m) \ \ {\tt FROM} \  D\  {\tt WHERE \ \ x\ <\ 10\
GROUP \ \ BY} \ a \\
Q2 = &{\tt SELECT \ } a, f(m) \ \ {\tt FROM} \  D\  {\tt GROUP \ \ BY} \ a \\
Q3 = &{\tt SELECT \ } a, f(m), {\tt CASE\ IF\ x\ <\ 10\ THEN\ 1\ ELSE\ 0\
END}\\ 
&as\ g1,\ 1\ as\ g2 \ \ {\tt FROM} \ D\ {\tt GROUP \ \ BY} \ a,\ g1,\ g2
\end{align*}

\stitle {Parallel Query Execution}:
  We execute multiple view queries in parallel since we expect co-executing queries 
  to share buffer pool pages and reduce disk accesses times. 
  However, the precise number of parallel queries needs to be tuned to take into account 
  buffer pool contention, locking and cache line contention \cite{Postgres_wiki}. 
  %As a result, we must identify the optimal number of parallel queries for our workload.

% We can do much better if we can use a single scan to evaluate multiple views simultaneously.

% In fact, our ideal operator $\mathcal{O}$ would perform a single scan of the
% data and compute all query results in one pass to return the top-$k$ views.
% Current database systems do not support this functionality.
% A handful of systems have recently introduced the SQL GROUPING
% SETS\footnote{GROUPING SETS allow the simultaneous grouping of query results by
% multiple sets of attributes.} functionality that could be used to support this
% operation. However, the grouping sets operator will not scale to tables with 
% a large number of attributes due to the large size of intermediate results.
% size of the eventual result maintained in memory will be $2 \times a \times f
% \times m$ attributes, multiplied by the size of an aggregate distribution.
% Furthermore, as we show in Section \ref{sec:in_memory_execution_engine},
% evaluating all views for the entire dataset is unnecessary and inefficient. 
% We can achieve much better performance by aggressively pruning low-utility
% views on the fly.

% Without this ideal operator $\mathcal{O}$,
% our optimizations minimize table scans in two ways: (1) minimize the total number of
% queries by intelligently combining queries, and (2) reduce the total
% execution time by running queries in parallel. 
% Our DBMS-backed execution engine for \SeeDB\ is agnostic towards the
% particular DBMS used to execute queries.
% However, because of significant differences in the way that row stores and
% column stores organize data, some of the optimizations described below
% will be more powerful in row stores and may actually hurt performance in column
% stores. In our experimental evaluation (Section \ref{sec:experiments}), we study
% the relative advantages of each of our optimizations for row and column stores.
% The optimizations supported by \SeeDB\ are described below.

% \subsubsection {Combine Multiple Aggregates} 
% A large number of view queries have the same group-by attribute but different
% aggregation attributes. 
% Therefore, \SeeDB\ combines all view queries with the same
% group-by attribute into a single, combined view query. For instance, instead of executing
% queries for views $(a_1$, $m_1$, $f_1)$, $(a_1$, $m_2$, $f_2)$ \ldots $(a_1$, $m_k$, $f_k)$
% independently, we can combine the $n$ views into a single view represented by
% $(a_1, \{m_1, m_2\ldots m_k\}$, $\{f_1, f_2\ldots f_k\})$. We demonstrate later
% on (Section \ref{sec:expts_dbms_execution_engine})  that this optimization
% offers a speed-up roughly proportional to the number of measure attributes.

\stitle {Combine Multiple Group-bys}:
% Efficient cube materialization is a problem well studied in the OLAP literature.
% Since the views considered by \SeeDB\ can be thought of as projections of the OLAP
% cube, we can adapt efficient materialization techniques to our problem.
Our multiple aggregate optimization reduces the problem of evaluating
all views to that of evaluating a multi-aggregate group-by along each dimension 
attribute of a table.
Therefore, \SeeDB\ has to determine the optimal way of executing these multi-attribute
{\it group-by} queries.
Savvy readers will recognize this problem as being a special case of materializing
an OLAP cube~\cite{olap_cube}.
Views evaluated by \SeeDB\ are projections of the OLAP cube and we can adapt techniqes
from the cube literature~\cite{} to find the optimial execution strategy for these queries.
% There are few differences in the \SeeDB\ setting and traditional OLAP cubes: (1) OLAP
% cube materialization is performed inside the database giving more flexibility in the
% techniques used; in \SeeDB\ every evaluation corresponds to a separate query; (2) 
Specifically, we work with a similar problem formulation: given a space budget $\mathcal{S}$,
and estimates of sizes of views (i.e. number of rows in the result), determine the optimal
way to combine multiple single-attribute group-by queries into multi-attribute group-by queries.
If we know the number of distinct values for each attribute, we can estimate the
size of a view containing dimension attributes $a_i$ and $a_j$ as $|a_i|\times |a_j|$.
Our problem becomes:
\vspace{-5pt}
\begin{problem}[Optimal Grouping]
Given a set of dimension attributes $A$ = \{$a_1$\ldots$a_n$\}, divide the
dimension attributes in $A$ into groups $A_1, \ldots, A_l$ (where $A_i$ is some
subset of $A$ and $\bigcap A_i$=$A$) such that if a query $Q$ groups the table by $A_i$, 
the total space budget for $Q$ does not exceed $\mathcal{S}$.
\vspace{-5pt}
\end{problem}

Notice that the above problem is isomorphic to the NP-Hard {\em bin-packing} problem~\cite{garey}.
If we let each dimension attribute
$a_i$ correspond to an item in the bin-packing problem with weight $\log (|a_i|)$,
and set the bin size to be $\log \mathcal{S}$,
then packing items into bins is identical to finding groups $A_1, \ldots, A_l$,
such that the estimated size of any query result is below $\mathcal{S}$.
We use the standard first-fit algorithm~\cite{first-fit} to find the optimal
grouping of dimension attributes.

% The first-fit algorithm works as follows:
% on initialization, all groups $A_1, \ldots, $ $A_l$ are empty.
%   For each dimension attribute, considered in an arbitrary order, place it in the first group
%   $A_i$ that it can ``fit into'', i.e., the worst case
%   number of distinct values for that $A_i < \tau$.
%   For bin-packing, this algorithm has a guarantee of using up to 1.7X more 
%   bins than necessary: here, this translates to up to 1.7X more groups than necessary.
% Notice that since this grouping is independent of the input query, we can
% perform grouping offline using more computationally expensive techniques as
% well.


% since we now store the aggregate for every combination
% of $a_1, a_2, \ldots, a_n$, 
% the number of aggregates that need to be recorded 
% is the number of distinct combinations
% of $(a_1, \ldots, a_n)$, which, in the worst case, 
% is proportional to $\prod_i |a_i|$.
% Thus, the number of aggregates that need to be recorded 
% grows exponentially (in the worst case) in the
% number of group-by attributes. 
% We will show (Section \ref{sec:experiments}) that 
% the time to execute a query with multiple group-by attributes does indeed
% depend on the number of distinct values present in the 
% combination of dimension attributes. 
% From a DBMS perspective, this is expected because keeping track of a large
% number of aggregates impacts computational time (e.g. for sorting in sort-based aggregate)
% as well as temporary storage requirements (e.g. for hashing in hash-based
% aggregate) making this technique ineffective for large numbers of
% distinct values.

% Consequently, when we combine group-by attributes, we must ensure that the
% number of distinct values remains {\it small enough}, below a specific 
% threshold $\tau$, that we determine based on system parameters.
% For simplicity, we ignore the correlations between dimension attribute values,
% and work with worst-case estimates. 
% The upper limit on the number of distinct values for a given combination of
% group-by attributes is given by the product of the number of distinct values
% for each attribute.
% For example, if we combine three dimension attributes $a_i$, $a_j$ and $a_k$
% with $|a_i|$, $|a_j|$ and $|a_k|$ distinct values respectively, the maximum number of
% distinct groups is $|a_i|\times |a_j| \times |a_k|$.
 % The number of distinct groups in turn depends on the correlation between
 % values of attributes that are being grouped together. 
 % For instance, if two
 % dimension attributes $a_i$ and $a_j$ have $n_i$ and $n_j$ distinct values
 % respectively and a correlation coefficient of $c$, the number of distinct
 % groups when grouping by both $a_i$ and $a_j$ can be approximated by
 % $n_i$$\ast$$n_j$$\ast$(1-$c$) for $c$$\neq$1 and $n_i$ for $c$=1 ($n_i$ must
 % be equal to $n_j$ in this case).
  
% Therefore, our problem can be stated as follows:
% \begin{problem}[Optimal Grouping Optimization]
% Given a set of dimension attributes $A$ = \{$a_1$\ldots$a_n$\}, divide the
% dimension attributes in $A$ into groups $A_1, \ldots, A_l$ (where $A_i$ is some
% subset of $A$ and $\bigcap A_i$=$A$) such that the worst-case number of distinct values
% for each group is below $\tau$.
% \vspace{-5pt}
% \end{problem}

\stitle{Other Optimizations}: 
To further speedup processing, we can pre-compute results for 
static views (e.g. comparison views on full tables) or operate on
pre-computed data samples. 
While we acknowledge these optimizations, we focus instead on techniques
that address the core problem of rapidly evaluating a large number of views.
Any of the above optimizations can take advantage of pre-computed results or 
samples.
%do not address the more important challenge of evaluating a large number of views.

%In Section \ref{XXX} we show that this optimization gives us a speedup of XXX.

% Given that the problem is NP-Hard, we use two strategies to group dimension attributes;
% the first strategy is adapted from the standard first-fit approximation algorithm~\cite{first-fit}
% for bin-packing; the second strategy is adapted from Huffman coding~\cite{huffman-codes}.
  
% \squishlist
% \item {\bf First-Fit}: The algorithm is simple;
% at first, all groups $A_1, \ldots, $ $A_l$ are empty.
% For each dimension attribute, considered in an arbitrary order, place it in the first group
% $A_i$ that it can ``fit into'', i.e., the worst case
% number of distinct values for that $A_i < \tau$.
% For bin-packing, this algorithm has a guarantee of using up to 1.7X more
% bins than necessary: here, this translates to up to 1.7X more groups than necessary.
  % In this trategy, we set an upper limit $V$ on the
  % number of distinct groups that any combination of dimension attributes can
  % produce.
  % Now, suppose that each dimension attribute $a_i$ $\in A$ has $n_i$ distinct
  % values.
  % We cast the problem of finding the optimal grouping of dimension attributes as
  % a version of bin-packing.
  % Let each dimension attribute $a_i$ be an item with volume $log(n_i)$ and
  % suppose that we are given bins of volume $log(V)$.
  % Then finding the optimal grouping of attributes is exactly equivalent to
  % finding the optimal packing of attribute items into the minimum number of
  % bins.
%   \item {\bf Huffman-Grouping}: Here, the algorithm works as follows:
%   we start by assigning each dimension attribute its own group,
%   and maintain these groups in sorted ascending order.
%   At each step, we take the two smallest groups out of this sorted list, combine
%   the dimension attributes in both of them into one new group, and then
%   add this new group to the mix and re-sort.
%   We keep doing this until we cannot combine the two smallest groups any more
%   without violating the threshold $\tau$.
% \squishend
% If this was vanilla bin-packing, the first-fit algorithm would be sufficient to
% find near-optimal groupings. 
% The advantage of having the huffman-grouping algorithm is the following: Instead
% of having a hard threshold $\tau$ on the worst-case number of distinct groups,
% the huffman-grouping algorithm can also adapt to the case where we
% are instead provided a black-box function that, given a particular
% grouping of attributes, returns a value for the ``goodness'' of the combination.
% As we will see in Section~\ref{sec:analytical_model}, 
% we develop an analytical model for the performance (i.e., the execution time)
% of the DBMS-backed execution engine. 
% This model can be used to derive this black-box function.
% Given a black-box function, the huffman-grouping algorithm operates
% as follows: until the ``goodness'' of the grouping keeps increasing 
% (i.e., the predicted execution time keeps decreasing),
% repeatedly combine the two smallest groups into a single group.
% We experimentally evaluate both strategies in Section \ref{sec:experiments}.
% 
% \srm{In 6.2.2., we mention the model.    Are we keeping this?  I think it could be ok to put the details in the appendix if we can show one graph that illustrates that the model can be used to predict how many groups to choose, etc.}

% \subsubsection{Combine target and comparison view query}
% \label{subsec:target_comparison_view}
% Since the target view and comparison views only differ in the subset of data
% that the query is executed on, we can easily rewrite these two view queries as
% one. For instance, for the target and comparison view queries $Q1$ and $Q2$
% shown below, we can add a group by clause to combine the two queries into $Q3$.
% \begin{align*} 
% Q1 = &{\tt SELECT \ } a, f(m) \ \ {\tt FROM} \  D\  {\tt WHERE \ \ x\ <\ 10\
% GROUP \ \ BY} \ a \\
% Q2 = &{\tt SELECT \ } a, f(m) \ \ {\tt FROM} \  D\  {\tt GROUP \ \ BY} \ a \\
% Q3 = &{\tt SELECT \ } a, f(m), {\tt CASE\ IF\ x\ <\ 10\ THEN\ 1\ ELSE\ 0\
% END}\\ 
% &as\ g1,\ 1\ as\ g2 \ \ {\tt FROM} \ D\ {\tt GROUP \ \ BY} \ a,\ g1,\ g2
% \end{align*}
% This rewriting allows us to obtain results for both queries in a single table
% scan. The impact of this optimization depends on the selectivity of the
% input query and the presence of indexes. When the input query is less selective,
% the query executor must do more work if the two queries are run separately. In
% contrast, in the presence of an index, running selective queries independently
% may be faster.
% \srm{In 4.2.3., I don't understand why the combined query wouldn't always be faster.  It seems like you are replacing two scans with one, which should just be better.  How could it not be (unless there are very selective filter predicates that aren't show in the example???)}
% \agp{I agree.}

% \subsubsection {Combine Multiple Group-bys}
% \label{subsec:mult_gb}
%   Similar to the multiple aggregates optimization, another optimization
%   supported by \SeeDB\ is to combine multiple queries with different group-by
%   attributes into a single query with multiple group-by attributes.
%   For instance, instead of executing queries for views $(a_1$, $m_1$, $f_1)$,
%   $(a_2$, $m_1$, $f_1)$ \ldots $(a_n$, $m_1$, $f_1)$ independently, we can
%   combine the $n$ views into a single view represented by $(\{a_1, a_2\ldots
%   a_n\}$, $m_1$, $f_1)$ and post-process results.

% Unlike the previous optimization, where the speed-up is proportional to the
% number of view queries combined, in this case, the situation is not as straightforward. 
% The reason for this is the following:
% since we now store the aggregate for every combination
% of $a_1, a_2, \ldots, a_n$, 
% the number of aggregates that need to be recorded 
% is the number of distinct combinations
% of $(a_1, \ldots, a_n)$, which, in the worst case, 
% is proportional to $\prod_i |a_i|$.
% Thus, the number of aggregates that need to be recorded 
% grows exponentially (in the worst case) in the
% number of group-by attributes. 
% We will show (Section \ref{sec:experiments}) that 
% the time to execute a query with multiple group-by attributes does indeed
% depend on the number of distinct values present in the 
% combination of dimension attributes. 
% From a DBMS perspective, this is expected because keeping track of a large
% number of aggregates impacts computational time (e.g. for sorting in sort-based aggregate)
% as well as temporary storage requirements (e.g. for hashing in hash-based
% aggregate) making this technique ineffective for large numbers of
% distinct values.

% Consequently, when we combine group-by attributes, we must ensure that the
% number of distinct values remains {\it small enough}, below a specific 
% threshold $\tau$, that we determine based on system parameters.
% For simplicity, we ignore the correlations between dimension attribute values,
% and work with worst-case estimates. 
% The upper limit on the number of distinct values for a given combination of
% group-by attributes is given by the product of the number of distinct values
% for each attribute.
% For example, if we combine three dimension attributes $a_i$, $a_j$ and $a_k$
% with $|a_i|$, $|a_j|$ and $|a_k|$ distinct values respectively, the maximum number of
% distinct groups is $|a_i|\times |a_j| \times |a_k|$.
%  % The number of distinct groups in turn depends on the correlation between
%  % values of attributes that are being grouped together. 
%  % For instance, if two
%  % dimension attributes $a_i$ and $a_j$ have $n_i$ and $n_j$ distinct values
%  % respectively and a correlation coefficient of $c$, the number of distinct
%  % groups when grouping by both $a_i$ and $a_j$ can be approximated by
%  % $n_i$$\ast$$n_j$$\ast$(1-$c$) for $c$$\neq$1 and $n_i$ for $c$=1 ($n_i$ must
%  % be equal to $n_j$ in this case).
  
% Therefore, our problem can be stated as follows:
% \begin{problem}[Optimal Grouping Optimization]
% Given a set of dimension attributes $A$ = \{$a_1$\ldots$a_n$\}, divide the
% dimension attributes in $A$ into groups $A_1, \ldots, A_l$ (where $A_i$ is some
% subset of $A$ and $\bigcap A_i$=$A$) such that the worst-case number of distinct values
% for each group is below $\tau$.
% \vspace{-5pt}
% \end{problem}
% Notice that the problem as stated above is isomorphic to the NP-Hard
% {\em bin-packing} problem~\cite{garey}: to see this, we let each dimension attribute
% $a_i$ correspond to an item in the bin-packing problem with weight $\log (|a_i|)$,
% and we set the threshold on the bin size to be $\log \tau$,
% then packing items into bins is identical to finding groups $A_1, \ldots, A_l$,
% such that the worst-case number of distinct values is below $\tau$.
% Thus, the problem as stated above is NP-Hard.

% We use the standard first-fit algorithm~\cite{first-fit} to find the optimal
% grouping of dimension attributes.
% The first-fit algorithm works as follows:
% on initialization, all groups $A_1, \ldots, $ $A_l$ are empty.
%   For each dimension attribute, considered in an arbitrary order, place it in the first group
%   $A_i$ that it can ``fit into'', i.e., the worst case
%   number of distinct values for that $A_i < \tau$.
%   For bin-packing, this algorithm has a guarantee of using up to 1.7X more 
%   bins than necessary: here, this translates to up to 1.7X more groups than necessary.
% Notice that since this grouping is independent of the input query, we can
% perform grouping offline using more computationally expensive techniques as
% well.

  % \subsubsection {Parallel Query Execution}
  % \label{subsec:parallel_exec}
  % While the above optimizations reduce the number of queries executed, we can
  % further speedup \SeeDB\ processing by executing view queries in parallel. When
  % executing queries in parallel, we expect co-executing queries to share pages in the
  % buffer pool for scans of the same table, thus reducing disk accesses and
  % therefore the total execution time. 
  % However, a large number of parallel queries can lead to poor performance for
  % several reasons including buffer pool contention, locking and cache line
  % contention \cite{Postgres_wiki}. 
  % As a result, we must identify the optimal number of parallel queries for our workload.
  
  % We do observe a reduction in the
  %overall latency when a small number of queries are executing in parallel;
  % however, the advantages disappear for larger number of queries running in
  % parallel. We discuss this further in the evaluation subsection.
% \subsubsection{Other Optimizations}
% To further speedup \SeeDB we can pre-compute various partial results.
% For example, in the case where our comparison view is constructed from the
%   entire underlying table (Example 1 in Section \ref{sec:introduction}),
%   comparison views are the same irrespective of the input query.
% Therefore, we can precompute all possible comparison views once and store
%   them for future use. 
%   Similarly, if we precompute a sample of the table being queried, we can run
% all view queries against this table to pick the top-$k$ views and only evaluate
% those views on the entire dataset.
% We do not experimentally evaluate these optimizations because while these
% problems reduce the computation required by \SeeDB, the
% challenges related to evaluating a large number of views (whether limited to
% target views or limited to a sample) remain unaddressed.
%   comparisons. If the dataset has $a$ dimension and
%   $m$ measure attributes, 
%   pre-computing comparison views would add $a \times m$
%   tables. This corresponds to an extra storage of $O(a\times m \times n \times f)$ where $n$
%   is the maximum number of distinct values in any of the $a$ attributes,
%   and $f$ is the number of aggregation functions. 
%   Note that pre-computation cannot be used in situations where the comparison
%   view depends on the target view (Example 2) or is directly specified by the
%   user (Example 3).
%   


