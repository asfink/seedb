%!TEX root=document.tex

\section{Incorporating Custom Engine into a DBMS}
\label{sec:incorporating}

In an ideal solution, we would incorporate the algorithms used in our custom
engine into a DBMS. 
The advantages of integrating into a DBMS are clear: we could take advantage of
the superior data storage and retrieval techniques in databases, and we would
avoid building a one-off, specialized system.
However, as the current database API stands, there is no way to share table
scans between operations, keep track of custom statistics or intercept the scan
mid-way to perform pruning.
If we are two implement a \VizRecDB-style operator in a traditional database
system, we would have to support the following operations:
\squishlist
\item Ability to store custom state during aggregation operations (e.g.
distributions corresponding to each view)
\item Ability to compute custom statistics during a table scan (in our case
utilities, their cumulative means and variances)
\item Ability to compute multiple aggregates and group-bys at the same time
(similar to the GROUPING SETS functionality)
\item Ability to intercept the scan periodically (think of a call back or a
sleep functionality) in order to update the custom state
\squishend

These operators can be embedded in the query executor or as part of a UDF.
We lay down these requirements because we expect that not just \VizRecDB\ but
other kinds of similar workloads would also benefit from such an API.
This implementation is out of scope for the current work, but we plan to explore
it in future work.