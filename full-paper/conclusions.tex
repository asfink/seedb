%!TEX root=document.tex

\section{Conclusions}
\label{sec:conc}

Finding the right visualization given a query of interest is a
laborious and time-consuming task.
In this paper, we presented \VizRecDB, a system to help data scientists 
rapidly identify possible visualizations of their data by finding
attributes that show the most deviation in comparison to an underlying data set.
We presented two implementations of \VizRecDB, one that runs on top of existing
DBMSs and another that is a custom execution engine that supports shared table scans
and aggressive pruning of low-quality views.
Our experimental evaluation on a range of real and synthetic datasets shows that
our optimizations reduce latency to just a few seconds to evaluate hundreds of different
visualizations.
In addition, our experiments on our custom execution engine show that our pruning
heuristics can reduce latency by 10-fold by aggresively pruning low-utility views.
This provides us a means to rapdily surface the top few views and then
gradually return additional views.
Finally, we demonstrated that our pruning
strategies do not adversely affect accuracy of views returned.