\section{View Pruning}
\label{sec:view_pruning}
Next, the View Generator computes pairwise correlations between subsets of the
entire set of possible views.
To do so, the View Generator computes the aggregate distributions for all views
over the entire data (i.e. the comparison view). 
Once the distributions for all views have been computed, it calculates the
correlation between distributions of views that contain dimension attributes
with the same size (i.e. number of distinct values).
Specifically, since the distributions are numeric, it computes the pairwise
Pearson correlation between the distributions.
Next, it (conceptually) builds a graph of all views. 
The nodes of this graph correspond to views and an edge exists between two nodes
if the correlation between the two views is greater than a threshold (set to 0.95 in
our experiments). 
In this graph, the View Generator then identifies cliques (and almost cliques).
Observe that a clique in this graph is a set of highly correlated views,
and therefore, these views are likely to have similar utility. 
As a result, we select a
single node from each clique as a representative view for that clique and
prune the remaining views. 
We can think of this procedure as clustering views based on similarity and
choosing a representative view form each cluster.
At the end of the offline step, the View Generator stores the list of
viable views that must be evaluated at run time.

When the View Generator is invoked at runtime, it reads the list of viable
views for the table, prunes them further based on the input query (e.g.
attributes present in the where clause of the query should not be present in
any view) and passes the remaining views to the Execution Engine for
evaluation.

In real datasets, we find that the offline processing significantly
reduces the number of views that must be evaluated. 
For example, in the
diabetes dataset discussed in Section \ref{sec:experiments}, offline pruning
reduces the total number of views from XXX to XXX due to our pruning
strategy. Similarly, for the banking dataset, the total views are
reduced from XXX to XXX. In Figure \ref{}, we show graphs generated by the
View Generator for both datasets.