%!TEX root=document.tex


\section{Related Work}
\label{sec:related_work}
\VizRecDB\ is related to work from multiple areas;
we review papers in each of the areas, and how they relate to
\VizRecDB. 

\stitle{Interactive Data Visualization Tools}:
Over the past few years, the research community has introduced a number of
interactive data analytics tools such as ShowMe, Polaris, and 
Tableau~\cite{DBLP:journals/cacm/StolteTH08, DBLP:journals/tvcg/MackinlayHS07}.
Similar visualization specification tools have also been introduced by the
database community, including Fusion
Tables~\cite{DBLP:conf/sigmod/GonzalezHJLMSSG10} and the
Devise~\cite{DBLP:conf/sigmod/LivnyRBCDLMW97} toolkit. 
Unlike \VizRecDB, which recommends visualizations automatically, these tools place
the onus on the analyst to specify the visualization to be generated.
For datasets with a large number of attributes, it is not possible
for the analyst to manually study all the attributes; hence, interactive
visualization needs to be augmented with automated visualization techniques.

A few recent systems have attempted to automate some aspects of data analysis
and visualization. Profiler is one such automated tool that allows analysts to
detect anomalies in data \cite{DBLP:conf/AVI/KandelPPHH12}.
% Our work is also similar to VizDeck which is a tool that given a dataset, uses a
% set of pre-determined rules to create diverse visualizations and
% allows the user to pick and choose the visualizations that seem relevant
% \cite{DBLP:conf/sigmod/KeyHPA12}.
% Thus, while powerful, VizDeck requires much more manual input than \VizRecDB. 
% In addition, the visualizations generated by VizDeck do not leverage the
% context of the underlying dataset, making the visualizations generated by
% both systems very different in flavor. 
% It would be instructive to augment
% VizDeck visualizations with \VizRecDB\ visualizations to study their relative
% utility.
Another related tool 
is VizDeck~\cite{DBLP:conf/sigmod/KeyHPA12}, which, given a dataset,
depicts all possible 2-D visualizations on a dashboard that the user can
manipulate by reordering or pinning visualizations.
Given that VizDeck generates all visualizations, it is only meant for 
small datasets; additionally, the VizDeck does not discuss techniques
to speed-up the generation of these visualizations. 

Statistical analysis and graphing packages such as R, SAS and Matlab could also
be used generate visualizations, but they lack the ability to filter and
recommend ``interesting'' visualizations. 

There has been some recent work on
scalable visualizations in the information visualization community. Immens~\cite{2013-immens} and Profiler (mentioned above) maintain a data cube in memory and use it to support rapid user interactions. While this approach is possible when the dimensionality
and cardinality is small (e.g., simple map visualizations of a single
attribute) it cannot be used with large tables and ad-hoc queries.
Pre-computation and pre-fetching are two other techniques that have
been used for scalability, e.g., \cite{hotmap} uses
precomputed image tiles for geographic visualization,
\cite{doshi2003prefetching} uses extensive pre-fetching and caching. 
A recent paper \cite{2014-viz-latency} discusses how high
latency in visualization systems reduces the rate at which users observe,
analyze and draw conclusions from data, thus making a strong case for
interactive response times.


% \agpneutral{Other recent work has addressed other aspects of visualization
% scalability, including prefetching and caching~\cite{doshi2003prefetching}, data
% reduction~\cite{burtini2013time} leveraging time series data
% mining~\cite{esling2012time}, clustering and
% sorting~\cite{guo2003coordinating,seo2005rank}, and dimension
% reduction~\cite{Yang:2003:VHD:769922.769924}. These techniques are orthogonal to
% our work, which focuses on speeding up the computation of a single visualization
% online.}\\

\stitle{Data Cubes:}
The work done in \VizRecDB\ is similar to previous literature in
browsing OLAP data cubes. 
Instead of building complete data cubes,
one can think of \VizRecDB\ views as projections of the cube along various
dimensions.
 Data cubes have been very well studied in the literature
\cite{DBLP:conf/SIGMOD/HarinarayanRU96, DBLP:jounral/DMKD/GrayCBLR97}, and work such as
~\cite{DBLP:conf/vldb/Sarawagi99, DBLP:conf/vldb/SatheS01,
DBLP:conf/vldb/Sarawagi00, DBLP:conf/SIGKDD/OrdonezC09} has explored the
questions of allowing analysts to find explanations for trends, get suggest for
cubes to visit, identify generalizations or patterns starting from a single
cube. 
This literature is not directly applicable to our problem since the cubes we
are considering have 10s to 100s of dimensions, making traditional cube
algorithms infeasible. 

% \stitle{General Purpose Data Analysis Tools:}
% Our work is also related to data mining and the work on building general purpose
% data analysis tools on top of databases. 
% For example, MADLib \cite{DBLP:conf/VLDB/HellersteinRSWF12}
% implements various analytic functions inside the database. 
% MLBase similarly
% \cite{DBLP:conf/CIDR/KraskaTDGFJ2013} provides a platform that allows users to
% run various machine learning algorithms on top of the Spark system
% \cite{DBLP:conf/SCC/ZahariaCFSS10}.



\stitle{Other:}
The techniques we use in our custom implementation of \VizRecDB\ draw upon work
from top-k ranking, statistical sampling and the multi-armed bandit strategies. 
In particular, the confidence interval technique discussed in Section
\ref{sec:confidence_interval} draws inspiration from the seminal top-k ranking
work by Fagin and others in \cite{DBLP:conf/pods/FaginLN01, DBLP:conf/vldb/IlyasAE04}.
Similarly, multi-armed bandits (referenced in Section
\ref{sec:multi_armed_bandit}) form 
a rich area of research having applications from ad auctions to reinforcement learning. 
Our technique is related to the original UCB algorithm \cite{AuerCF02, LaiR85}
as well as recent work related to the top-$k$ MAB variant \cite{BubeckWV13,
audibert2010best}.
Finally, finding interesting visualizations in data also involves understanding
user preferences. 
In future work, we plan to learn user preference models towards visualizations
using techniques similar to \cite{CHI:YangLZ14, IUIGotzW09}. 

