%!TEX root=document.tex


\section{Related Work}
\label{sec:related_work}
\SeeDB\ is related to work from multiple areas;
we review papers in each of the areas, and describe how they relate to
\SeeDB. 

\stitle{Visualization Tools}:
The visualization 
research community has introduced a number of
visual analytics tools such as ShowMe, Polaris, Spotfire and 
Tableau~\cite{DBLP:journals/cacm/StolteTH08, DBLP:journals/tvcg/MackinlayHS07,spotfire}.
In these tools, the goal is to make it simpler for analysts 
to pose specifications of what they would like to examine
(e.g., they would like to examine sales by year for cars),
and the visual analytics tool would provide an appropriate
visualization medium for this visualization specification
(e.g., a trend line or a bar chart).
Most of the ``smarts'' in these tools is restricted
to a set of aesthetic rules of thumb that
guide which visualization is appropriate
given a visual specification.
Similar visual specification tools 
have also been introduced by the
database community, including Fusion
Tables~\cite{DBLP:conf/sigmod/GonzalezHJLMSSG10} and the
Devise~\cite{DBLP:conf/sigmod/LivnyRBCDLMW97} toolkit. 

In all these tools, the onus is on the analyst to specify
which visualization they would like to generate.
For datasets with a large number of attributes,
it is often hard for the analyst to manually study
every single attribute.
Hence, in \SeeDB, our goal is to support powerful
automated visualization recommendation functionality
based on a generalized distance function,
in addition to the simple visualization
specification functionalities described above,
which is more appropriate when the analyst already
knows which visualizations they should be examining.

Statistical analysis and graphing packages such as R, SAS and Matlab could also
be used generate visualizations, but they lack the ability to filter and
recommend visualizations. 


\stitle{Partial Automation.}
A few recent systems have attempted to automate some aspects of data analysis
and visualization. Profiler is one such automated tool that allows analysts to
detect anomalies in data \cite{DBLP:conf/AVI/KandelPPHH12}.
Profiler provides some visualization recommendation functionality,
but is restricted to the determination of the binning 
for the $x$ axis: in particular, it decides which granularity
is appropriate to bin on to depict the most interesting relationships 
between data. 
Since this is a much simpler (and less ambitious) problem
than ours, sophisticated techniques are not necessary.
% Our work is also similar to VizDeck which is a tool that given a dataset, uses a
% set of pre-determined rules to create diverse visualizations and
% allows the user to pick and choose the visualizations that seem relevant
% \cite{DBLP:conf/sigmod/KeyHPA12}.
% Thus, while powerful, VizDeck requires much more manual input than \SeeDB. 
% In addition, the visualizations generated by VizDeck do not leverage the
% context of the underlying dataset, making the visualizations generated by
% both systems very different in flavor. 
% It would be instructive to augment
% VizDeck visualizations with \SeeDB\ visualizations to study their relative
% utility.
Another related tool 
is VizDeck~\cite{DBLP:conf/sigmod/KeyHPA12}, which, given a dataset,
depicts all possible 2-D visualizations on a dashboard that the user can
manipulate by reordering or pinning visualizations.
Given that VizDeck generates all visualizations, it is only meant for 
small datasets; additionally, the VizDeck does not discuss techniques
to speed-up the generation of these visualizations. 


\stitle{Scalable Visualizations.} 
There has been some recent work on
scalable visualizations in the information visualization community. Immens~\cite{2013-immens} and Profiler (mentioned above) maintain a data cube in memory and use it to support rapid user interactions. While this approach is possible when the dimensionality
and cardinality is small (e.g., simple map visualizations of a single
attribute) it cannot be used with large tables and ad-hoc queries.
Pre-computation and pre-fetching are two other techniques that have
been used for scalability, e.g., \cite{hotmap} uses
precomputed image tiles for geographic visualization,
\cite{doshi2003prefetching} uses extensive pre-fetching and caching. 
Our recent paper describes techniques for generating visualizations
approximately, but with ordering guarantees~\cite{DBLP:journals/corr/KimBPIMR14};
these techniques could be easily employed in our setting to further
improve the response times.
A recent paper \cite{2014-viz-latency} discusses how high
latency in visualization systems reduces the rate at which users observe,
analyze and draw conclusions from data, thus making a strong case for
interactive response times.

Finally, finding interesting visualizations in data also involves understanding
user preferences. 
In future work, we plan to learn user preference models towards visualizations
using techniques similar to \cite{CHI:YangLZ14, IUIGotzW09}. 


% \agpneutral{Other recent work has addressed other aspects of visualization
% scalability, including prefetching and caching~\cite{doshi2003prefetching}, data
% reduction~\cite{burtini2013time} leveraging time series data
% mining~\cite{esling2012time}, clustering and
% sorting~\cite{guo2003coordinating,seo2005rank}, and dimension
% reduction~\cite{Yang:2003:VHD:769922.769924}. These techniques are orthogonal to
% our work, which focuses on speeding up the computation of a single visualization
% online.}\\

\stitle{Data Cube Materialization:} 
OLAP (Online Analytical Processing)~\cite{olap} 
concerns itself with the 
computation and materialization of all possible 
{\em data cubes}~\cite{DBLP:jounral/DMKD/GrayCBLR97},
considering all possible groupings of attributes,
at different granularities, and all possible aggregations of other attributes.
These data cubes are used for report generation for Business Intelligence
(BI) applications.
Often, even when the number of attributes and number
of distinct values of each attribute is relatively small,
the space required to materialize and store these data cubes is
very large --- as a result of this, in practice, data cubes
are only stored, if at all, for a small number of attributes 
(typically less than a handful).
There has been some work on identifying, given a query workload,
which cubes to materialize within a certain storage budget,
so as to minimize the amount of work to be performed when a
query is provided~\cite{DBLP:conf/VLDB/AgarwalADG96,DBLP:conf/SIGMOD/HarinarayanRU96}.
In our setting, since the number of attributes can be rather large,
and since any ad-hoc data exploration queries can be provided on-the-fly, materialization
is not an option except for the most common or popular attributes.

That said, the optimization techniques underlying cube materialization 
is similar in spirit to our batching optimizations in Section~\ref{XXX}
(but not similar to our pruning optimizations).
However, the underlying techniques are different from ours in three ways:
(a) Cube materialization concerns itself with the computation
of these aggregates offline, as opposed to online: the bottleneck
is not computation, but storage.
(b) In our case, we definitely need to compute the results for
all possible views; as a result we create several batches,
each of which fit within our resource constraints. 
In cube materialization there is a single ``batch'' of aggregates
that is computed offline (rather than multiple batches) and then stored.
(c) Furthermore, there can be several cubes that are simply not materialized---queries
that refer to these cubes can certainly make a pass on the entire dataset if need be.
In our case, we cannot afford to not compute some views.



\stitle{Browsing Data Cubes:}
There has been some limited work on using data mining techniques
to aid in the exploration of the 
cube~\cite{DBLP:conf/vldb/Sarawagi99, DBLP:conf/vldb/SatheS01, DBLP:conf/vldb/Sarawagi00, 
DBLP:conf/SIGKDD/OrdonezC09}
For instance, Sarawagi et al.~\cite{DBLP:conf/EDBT/SarawagiAM98, DBLP:conf/vldb/Sarawagi00} 
have explored 
the question of finding ``interesting'' cells in a cube.
The interestingness of a cell is defined by how surprising
its value is given the other values in the cube:
\cite{DBLP:conf/EDBT/SarawagiAM98} uses techniques 
based on a table analysis method while
\cite{DBLP:conf/vldb/Sarawagi00} uses techniques based on entropy to find interesting cells.
Inherent interestingness is a much simpler problem than the
one addressed in our paper, since it can be precomputed before
any queries arrive. (This is because inherent interestingness is independent of queries.) 
Due to this, 
the techniques in browsing \cite{DBLP:conf/EDBT/SarawagiAM98, 
DBLP:conf/vldb/Sarawagi00} cannot be applied in our context.
Furthermore, instead of comparing individual cells, 
\SeeDB\ evaluates sets of aggregates (i.e. distributions
for our view), thus focusing on trends in values rather than individual values:
unlike individual cube values, trends can be more easily displayed
and examined on visual interfaces.

Another work by Sarawagi in this space (\cite{DBLP:conf/vldb/Sarawagi99}) 
proposes techniques to explain an
increase or decrease in a specific aggregate by drilling down into that aggregate.
Once again, while related, 
this work addresses a different question; \SeeDB\ seeks to find interesting
differences between two datasets rather than explaining known differences. 
\agp{The prev paragraph is quite weak. I'll need to look into the paper to identify the 
differences.}

% Second, instead of finding ``intrinsically'' interesting aggregates, 
% \SeeDB\ evaluates aggregate views with respect to a comparison dataset. 

% \agp{If we do have a general metric, get rid of sentence below.}
% We note however that incorporating an ``intrinsic'' interesting-ness metric into our utility function is an avenue for future work.


\stitle{Multi-Query Optimization:} Our goal of evaluating
many views in parallel is an instance of multi-query 
optimization~\cite{DBLP:journals/tods/Sellis88,DBLP:journals/pvldb/KementsietsidisNCV08,DBLP:journals/pvldb/WangC13},
i.e., optimizing the execution of a collection of queries rather
than one at a time.
The work on multi-query optimization focuses on the creation of ``global''
query plans for a collection of queries rather than a single one,
such that the intermediate operators and results are shared across queries.
A special case of multi-query optimization 
is the work on {\em shared scans}~\cite{DBLP:conf/VLDB/ZukowskiHNB07,DBLP:conf/sigmod/HollowayRSD07,DBLP:journals/pvldb/UnterbrunnerGAFK09,DBLP:conf/icde/RamanSQRDKNS08,DBLP:journals/pvldb/QiaoRRHL08},
where the goal is share scans to evaluate in parallel a collection of queries.
In our case, the problem is much simpler than both the shared scans work and 
the general multi-query optimization
framework, since we precisely know which candidate visualizations we need to evaluate,
given a query: this means we can batch them, potentially in advance of any
queries arriving in the system; these candidate visualizations are all simple
aggregation queries, and have predictable selectivities,
leading them to be more amenable to optimization tailored to our setting.
Furthermore, in our case, we can avoid evaluating some visualizations beyond a point
if we find that their utility is low---other shared scans and multi-query optimization
schemes do not have the luxury of not evaluating some visualizations.







% The work done in \SeeDB\ is similar to previous literature in
% browsing OLAP data cubes. 
% Instead of building complete data cubes,
% one can think of \SeeDB\ views as projections of the cube along various
% dimensions.
%  Data cubes have been very well studied in the literature
% \cite{DBLP:conf/SIGMOD/HarinarayanRU96, DBLP:jounral/DMKD/GrayCBLR97}, and work such as
% ~\cite{DBLP:conf/vldb/Sarawagi99, DBLP:conf/vldb/SatheS01,
% DBLP:conf/vldb/Sarawagi00, DBLP:conf/SIGKDD/OrdonezC09} has explored the
% questions of allowing analysts to find explanations for trends, get suggest for
% cubes to visit, identify generalizations or patterns starting from a single
% cube. 
% This literature is not directly applicable to our problem since the cubes we
% are considering have 10s to 100s of dimensions, making traditional cube
% algorithms infeasible. 


% \stitle{General Purpose Data Analysis Tools:}
% Our work is also related to data mining and the work on building general purpose
% data analysis tools on top of databases. 
% For example, MADLib \cite{DBLP:conf/VLDB/HellersteinRSWF12}
% implements various analytic functions inside the database. 
% MLBase similarly
% \cite{DBLP:conf/CIDR/KraskaTDGFJ2013} provides a platform that allows users to
% run various machine learning algorithms on top of the Spark system
% \cite{DBLP:conf/SCC/ZahariaCFSS10}.



\stitle{Other Related Work:}
The techniques we use in our custom implementation of \SeeDB\ is
related to work in top-k ranking, statistical sampling and the multi-armed bandit strategies. 

The confidence interval-based technique discussed in Section \ref{sec:confidence_interval} 
is similar to top-k based pruning algorithms developed 
in other contexts~\cite{DBLP:conf/pods/FaginLN01, 
DBLP:conf/vldb/IlyasAE04, DBLP:conf/ICDE/ReDS07},
and similar to our work on  
sampling for visualizations~\cite{DBLP:journals/corr/KimBPIMR14}.
The techniques in the former papers on top-$k$ pruning are 
tailored more towards picking individual desirable tuples,
rather than visualizations (which correspond to an array of aggregates),
and thus the techniques are very different.
The emphasis in the latter work is to ensure the generation of
approximate visualizations with certain guarantees---here,
our goal is to evaluate visualizations (possibly approximately). 

Similarly, multi-armed bandits (referenced in Section \ref{sec:multi_armed_bandit}) form 
a rich area of research having applications from ad auctions to reinforcement learning. 
Our problem presents a novel application of multi-armed bandit strategy to 
exploratory data analysis.
The technique we adopt is 
a variant of the original UCB algorithm \cite{AuerCF02, LaiR85},
as well as recent work related to the top-$k$ MAB variant \cite{BubeckWV13,
audibert2010best}. \mpv{revisit papers -- how is it related?}

